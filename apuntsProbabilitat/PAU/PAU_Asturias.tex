\item (2017 Juny Opció A P4) Una urna A contiene tres bolas numeradas del 1 al 3 y otra urna B, seis bolas numeradas del 1 al 6. Se elige, al azar, una urna y se extrae una bola.
\begin{enumerate}
 \item ¿Cuál es la probabilidad de que sea una bola con el número 1?
 \item Si extraída la bola resulta tener el número 1, ¿cuál es la probabilidad de que proceda de la urna A?
\end{enumerate}

\item (2017 Juny Opció B P4) En un asociación benéfica se reparten dos productos, harina y leche. Todas la personas que entran cogen dos unidades a elegir entre los dos tipos de producto. El 70\% de las personas que entran cogen harina y el 40\% los dos productos. Calcula:
\begin{enumerate}
 \item La probabilidad de que una persona que entre coja leche.
 \item La probabilidad de que una persona que entre coja un solo tipo de producto.
 \item Una persona que sale de la asociación lleva leche. ¿Cuál es la probabilidad de que haya cogido también harina?
\end{enumerate}

\item (2017 Set. Opció A P4) En una cierta enfermedad el 60\% de los pacientes son hombres y el resto mujeres. Con el tratamiento que se aplica se sabe que se curan un 70\% de los hombres y un 80\% de las mujeres. Se elige un paciente al azar.
\begin{enumerate}
 \item Calcula la probabilidad de que se cure de la enfermedad.
 \item Si un paciente no se ha curado, ¿cuál es la probabilidad de que sea mujer?
\end{enumerate}

\item (2017 Set. Opció B P4) De una baraja española Daniel y Olga extraen 8 cartas: los cuatro ases y los cuatro reyes. Con esas 8 cartas Olga da dos cartas a Daniel y posteriormente una para ella. Calcula:
\begin{enumerate}
 \item La probabilidad de que Daniel tenga dos ases.
 \item  La probabilidad de que Daniel tenga un as y un rey.
 \item  La probabilidad de que Olga tenga un as y Daniel no tenga dos reyes.
\end{enumerate}

\item (2018 Juny Opció A P4) En un espacio muestral se tienen dos sucesos independientes: $A$ y $B$. Se conocen las siguientes probabilidades: $p(A \cap B) = 0.3$ y $p(A|B) = 0.5$. Calcula
\begin{enumerate}
 \item $p(A)$ y $p(B)$.
 \item $p(A \cup B)$ y $p(B|A)$.
 \item La probabilidad de que no ocurra ni el suceso $A$ ni el suceso $B$.
\end{enumerate}

\item (2018 Juny Opció B P4) En la siguiente tabla se muestra la distribución de un grupo de personas en relación al consumo de tabaco:
\begin{center}
    \begin{tabular}{|c|c|c|}
            \hline
            & Fumador & No fumador \\ \hline
            Hombres & 10 & 20  \\ \hline
            Mujeres & 20 & 40  \\ \hline
    \end{tabular}
\end{center}
Se elige en ese grupo una persona al azar. Calcula las probabilidades de los siguientes sucesos diferentes:
\begin{enumerate}
 \item Sea fumador.
 \item Sabiendo que es fumador, se trate de una mujer.
 \item Se extrae una segunda persona al azar. ¿Cuál es la probabilidad de que una fume y la otra no?
\end{enumerate}

%  Format dificulta la còpia
% \item (2018 Set. Opció A P4)
% \begin{enumerate}
%  \item
%  \item
% \end{enumerate}

% \item (2018 Set. Opció B P4)
% \begin{enumerate}
%  \item
%  \item
% \end{enumerate}

\item (2019 Juny Opció A P4) Un monitor de tenis compra un cañón para lanzar bolas. En las especificaciones del cañón se indica que falla el lanzamiento el 10\% de la veces.
\begin{enumerate}
 \item ¿Cuál es la probabilidad de que, de 20 bolas lanzadas, se tengan exactamente 5 fallos?
 \item ¿Cuál es la probabilidad de que como mucho falle 2 veces de los 20 lanzamientos?
\end{enumerate}
Nota: Se pueden dejar indicadas las operaciones en potencias, sin necesidad de realizarlas.

\item (2019 Juny Opció B P4) Pedro y Luis son aficionados a los dardos. Pedro acierta en el centro el 10\% de las veces y cada vez que acierta gana 400\euro{}. Luis acierta en el centro el 20\% de las veces y cada vez que acierta gana 100\euro{}. Cuando fallan no ganan ni pierden nada. Tira cada uno dos dardos. Calcula las siguientes probabilidades:
\begin{enumerate}
 \item Que Luis acierte en el centro las dos veces.
 \item Que Pedro acierte en el centro una sola vez.
 \item Que entre los dos hayan ganado 600\euro{}.
\end{enumerate}

\item (2019 Set. Opció A P4) Alicia tiene dos cajones. En uno tiene las camisetas y en el otro las faldas. La tabla muestra el número de todas las prendas que guarda en los dos cajones agrupadas en tres tipos: lisas, dibujos o rayas.
\begin{center}
    \begin{tabular}{|c|c|c|c|}
            \hline
            & Lisas & Dibujos & Rayas \\ \hline
            Camisetas & 10 & 5 & 10 \\ \hline
            Faldas & 5 & 15 & 10  \\ \hline
    \end{tabular}
\end{center}
Se elige al azar una prenda de cada cajón. Calcula la probabilidad de que:
\begin{enumerate}
 \item Las dos sean de rayas.
 \item Las dos sean del mismo tipo.
 \item Al menos una de ellas no sea de rayas.
\end{enumerate}

% Estadśitica
% \item (2019 Set. Opció B P4)
% \begin{enumerate}
%  \item
%  \item
% \end{enumerate}

\item (2020 Juny B4.A) En un espacio muestral se tienen dos sucesos: A y B. Se conocen las siguientes probabilidades:\\
$P (A \cap B) = 0.3$, $P (A|B) = P (B|A)$ y $P (\overline{A}) = 0.2$ ( $\overline{A}$ suceso contrario). Calcula:
\begin{enumerate}
 \item $P (B|A).$
 \item $P (B).$
 \item ¿Son los sucesos independientes?
\end{enumerate}

%Estadística
% \item (2020 Juny B4.B)
% \begin{enumerate}
%  \item
%  \item
% \end{enumerate}

\item (2020 Set. B4.A) En un curso de un instituto hay tres clases: la clase A con 50 alumnos, la clase B con 30 y la clase C con 20. Cada clase tiene un profesor distinto de matemáticas. Con el profesor de la clase A aprueban el 40\% de los alumnos, con el de la clase B el 50\% y con el de la clase C el 75\% de los alumnos. Se coge al azar un alumno del curso. Calcula:
\begin{enumerate}
 \item La probabilidad de que el alumno haya aprobado matemáticas.
 \item Sabiendo que ha aprobado, cuál es la probabilidad de que sea de la clase B.
\end{enumerate}

%Estadística
% \item (2020 Set. B4.B)
% \begin{enumerate}
%  \item
%  \item
% \end{enumerate}

\item (2021 Juny B4.A) En un edificio hay dos ascensores. Cada vecino, cuando utiliza el ascensor, lo hace en el primero el 60\% de las veces y en el segundo el 40\%. El porcentaje de fallos del primer ascensor es del 3\% y del segundo es del 8\%.
\begin{enumerate}
 \item Un vecino usa un ascensor. ¿Cuál es la probabilidad de que el ascensor falle?
 \item Otro día, un vecino coge un ascensor y le falla. ¿Cuál es la probabilidad de que haya sido el
segundo?
\end{enumerate}

%Estadística
% \item (2021 Juny B4.B)
% \begin{enumerate}
%  \item
%  \item
% \end{enumerate}

\item (2021 Set. B4.A) Se tienen tres cajas. En la caja A hay 4 bolas negras y 6 bolas rojas. En la caja B, 6 dados negros y 2 dados rojos y en la caja C, 2 dados negros y 4 dados rojos. El suceso consiste en sacar una bola y un dado. En primer lugar se extrae al azar una bola de la caja A. Si es negra, se
extrae al azar un dado de la caja B pero, si la bola es roja se extrae al azar un dado de la caja C.
Calcula las probabilidades de los siguientes sucesos sin relación entre ellos:
\begin{enumerate}
 \item La probabilidad de que la bola y el dado sean rojos.
 \item La probabilidad de que la bola y el dado sean del mismo color.
 \item La probabilidad de que el dado sea rojo.
\end{enumerate}

%Estadística
% \item (2021 Set. B4.B)
% \begin{enumerate}
%  \item
%  \item
% \end{enumerate}

\item (2022 Juny B4.A) Se tienen tres sobres, A, B y C. En el sobre A hay dos cartas de copas y tres
de bastos. En el sobre B tres cartas de copas y dos de bastos y en el sobre C cuatro de copas y una de bastos. Se tira un dado y se saca una carta del sobre A si el resultado es impar, del sobre B si el resultado es 4 o 6 y del sobre C si el resultado es un 2.
\begin{enumerate}
 \item Calcula la probabilidad de que se obtenga una carta de bastos.
 \item Se extrae una carta y resulta ser copas ¿cuál es la probabilidad de que se haya extraído del sobre B?
\end{enumerate}

%Estadística
% \item (2022 Juny B4.B)
% \begin{enumerate}
%  \item
%  \item
% \end{enumerate}

\item (2022 Set. B4.A) En una oficina del ayuntamiento se asigna un número a cada persona que
entra. Se observa que el 70\% de las personas que entran son mujeres. El 40\% de los
hombres y el 30\% de las mujeres que entran son menores de 30 años.
\begin{enumerate}
 \item Calcule la probabilidad de que un número sea asignado a una persona menor de 30 años.
 \item ¿Cuál es la probabilidad de que un número sea asignado a un hombre que no tiene menos de 30 años?
 \item Si la persona a la que se le ha asignado un número no tiene menos de 30 años ¿cuál es la probabilidad de que sea hombre?
\end{enumerate}

%Estadística
% \item (2022 Set. B4.B)
% \begin{enumerate}
%  \item
%  \item
% \end{enumerate}

\item (2023 Juny P7) Una compañía tiene tres centrales en Europa en la que se fabrica el mismo producto. El 60\% de las unidades de dicho producto se fabrica en España, el 25\% en Francia y el resto en Portugal. Se observa que de las unidades fabricadas tienen algún defecto el 1\% de los fabricados en España, el 0.5\% de los fabricados en Francia y el 2\% de los fabricados en Portugal. El departamento de control de calidad central toma una de las unidades fabricadas al azar.
\begin{enumerate}
 \item ¿Cuál es la probabilidad de que la unidad seleccionada tenga algún defecto?
 \item Si la unidad seleccionada es defectuosa ¿cuál es la probabilidad de que haya sido fabricada en Portugal?
\end{enumerate}

%Estadística
% \item (2023 Juny P8)
% \begin{enumerate}
%  \item
%  \item
% \end{enumerate}

\item (2023 Set. P7) Una imprenta compra la tinta a dos empresas distintas. En la empresa A compra el 60\% de sus pedidos, y el resto a la empresa B. Se observa que el 1.6\% de las cajas de tinta de la empresa A llegan con defecto, mientras que de la empresa B sólo el 0.9\% son defectuosas. Se toma una caja al azar:
\begin{enumerate}
 \item Calcula la probabilidad de que la caja sea defectuosa.
 \item Si la caja seleccionada no es defectuosa, calcule la probabilidad de que se haya comprado a la empresa A.
\end{enumerate}

%Estadística
% \item (2023 Set. B8)
% \begin{enumerate}
%  \item
%  \item
% \end{enumerate}
