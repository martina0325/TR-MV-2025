
\item Un ordinador personal té operatius dos programes antivirus A1 i A2 que actuen simultàniament i de forma independent. Davant la presència d'un virus, el programa A1 el detecta amb una probabilitat de 0.9 i el programa A2 el detecta amb una probabilitat de 0.8. Calculeu de forma raonada:
\begin{enumerate}
    \item La probabilitat que un virus qualsevol sigui detectat.
    \item  Si un virus ha estat detectat, quina és la probabilitat que l'hagi detectat l'antivirus A1?
    \item Si un virus ha estat detectat, quina és la probabilitat que l'hagin detectat els dos antivirus A1 i A2?
    \item Un software addicional altera el funcionament de l'antivirus A2 de manera que la probabilitat que detecti un virus ja no és de 0.8. Quina és aquesta nova probabilitat si sabem que un virus és detectat per A1 i no per A2 amb probabilitat 0.27?
\end{enumerate}


\item En un poble hi ha dos instituts que anomenarem A1 i A2. En tots dos instituts es pot estudiar el batxillerat científic (que anomenarem B1) o l'humanístic (que anomenarem B2). Seleccionem un alumne a l'atzar i se sap que la probabilitat que pertanyi a l'institut A1 és de 0.3, la probabilitat que pertanyi a l'institut A2 és de 0.7. D'altra banda, la probabilitat que estudiï el batxillerat científic si sabem que pertany a l'institut A1 és de 0.55 mentre que la probabilitat que estudiï el batxillerat científic si sabem que pertany a l'institut A2 és de 0.59.
 \begin{enumerate}
    \item Calcula les probabilitats que un alumne estudiï el batxillerat B1 a l'institut A1, que estudiï el batxillerat B1 a l'institut A2, que estudiï el batxillerat B2 a l'institut A1, i que estudiï el batxillerat B2 a l'institut A2.
    \item Si en aquest poble hi ha exactament 1000 estudiants, quants esduïen cada batxillerat a cada institut?
    \item El curs vinent arribaran 20 alumnes nous al poble i tots faran batxillerat B2 a l'institut A1. Quina serà la nova probabilitat que un alumne estudiï batxillerat B1 si sabem que pertany a l'institut A1?
\end{enumerate}

\item Els components electrònics produïts per una determinada empresa són defectuosos amb una certa probabilitat $p$. L'empresa ven els components en paquets de 10 i es compromet a retornar els diners si el paquet conté 2 o més components defectuosos.
\begin{enumerate}
    \item Calcula, en funció de $p$, la probabilitat que et retornin els diners si compres un paquet de components.
    \item Si $p=0.01$, quina és la probabilitat de que, comprant 3 paquets de components, etretornin els diners de, com a mínim, un dels paquets? Aquest resultat augmenta o disminueix quan $p$ augmenta? Raona la resposta.
    \item Si $p=0.01$, calcula la probabilitat que comprant 4 paquets et retornin els diners d'exactament dos d'ells.
\end{enumerate}

\item Considera l'experiment següent: tirem un dau equilibrat i, a continuació, tirem tantes monedes (equilibrades també) com indiqui el resultat del dau.
\begin{enumerate}
    \item Calcula la probabilitat que obtinguem exactament 3 cares.
    \item Calcula la probabilitat que obtinguem exactament 3 cares sabent que el resultat del dau ha estat un nombre parell.
    \item Calcula la probabilitat que obtinguem exactament 3 cares sabent que la primera moneda ha donat creu.
\end{enumerate}

\item L'Anna i el Blai juguen al joc següent: començant per l'Anna, s'alternen tirant una moneda equilibrada fins a un màxim de 4 cops cadascú; el primer que obtingui cara guanya, i si els hi surten vuit creus empaten.
\begin{enumerate}
    \item Calcula la probabilitat que guanyi l'Anna i la probabilitat que guanyi el Blai. Qui té més possibilitats de guanyar?
    \item Aquestes dues quantitats han de sumar 1? Justifica la resposta.
    \item Ara suposem que la moneda està trucada i que la probabilitat que surti cara en una tirada és $0<p<1$. Quan ha de ser p per tal que l'Anna tingui el triple de possibilitats de guanyar el joc?
\end{enumerate}

\item Tirem un dau equilibrat repetides vegades fins que surti un sis, moment en el qual parem.
\begin{enumerate}
    \item Quina és la probabilitat que després de $n$ tirades encara no hagi sortit cap sis?
    \item Quantes tirades hem de fer, com a mínim, per tal que la probabilitat que surti un sis sigui igual o superior a 0.95?
    \item Sabent que ens ha sortit el primer sis a la cinquena tirada, quina és la probabilitat que no hagi sortit cap cinc ni cap quatre?
\end{enumerate}

